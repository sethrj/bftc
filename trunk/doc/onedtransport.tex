\documentclass[]{SRJcommon}
%%%%%%%%%%%%%%%%%%%%%%%%%%%%%%%%%%%
\title{One-dimensional computational transport}
\author{Seth R.~Johnson}
\date{2008/12/21}
%%%%%%%%%%%%%%%%%%%%%%%%%%%%%%%%%%%
\begin{document}
 \makeatletter
 \hypersetup{pdfauthor={\@author}, pdftitle={\@title}}
 \makeatother
%%%%%%%%%%%%%%%%%%%%%%%%%%%%%%%%%%%%%%%%%%%%%%%%%%%%%%%%%%%%%%%%%%%%%%%%%%%%%%%%
\maketitle
%%%%%%%%%%%%%%%%%%%%%%%%%%%%%%%%%%%%%%%%%%%%%%%%%%%%%%%%%%%%%%%%%%%%%%%%%%%%%%%%
\section{Exact one-group}
We consider the one-dimensional transport equation:
$$
\mu \oder{\psi}{x}( x, \mu) + \Sigma_t(x) \psi(x,\mu) 
= \int_{-1}^{1} \Sigma_{s}(\mu \mu') \psi(x,\mu')  \ud\mu' 
  + Q(x, \mu)
$$
or, writing the scattering in terms of Legendre polynomials, truncating to the
$L$th polynomial,
\begin{subequations}
  \label{eqs:exactwithbc}
\begin{equation}
\mu \pder{\psi}{x}( x, \mu) + \Sigma_t(x) \psi(x,\mu) 
= \sum_{l=0}^{L}  \frac{2l+1}{2} \Sigma_{sl}(x) P_{l}(\mu)
  \int_{-1}^{1} P_{l} (\mu') \psi(x,\mu')  \ud\mu' 
  + Q(x, \mu)\,,
  \label{eq:exact}
\end{equation}
for $0 \le x \le X$ and $-1 < \mu < 1$. This equation also has the boundary
conditions
\begin{align}
  \psi(0, \mu) &= \psi^L(\mu) \,, \quad 0 \le \mu < 1 
  \\
  \psi(X, \mu) &= \psi^R(\mu) \,, \quad -1 < \mu \le 0 \,.
\end{align}
\end{subequations}

%%%%%%%%%%%%%%%%%%%%%%%%%%%%%%%%%%%%%%%%%%%%%%%%%%%%%%%%%%%%%%%%%%%%%%%%%%%%%%%
\section{Discrete ordinates}
To formulate the discrete ordinates approximation, we let $\psi(\mu)$ only hold
on $N$ specific ordinate directions, each one with angle $\mu_n$ and weight
$w_n$. We always require $N$ to be an even number, and that $\mu_n$ increase
with increasing $n$. The weights satisfy
$\sum_{n=1}^{N} w_n = 2$, analogous to how $\int_{-1}^{1} \ud \mu = 1$. 
The angular flux along each direction is $\psi(\mu_n) = \psi_n$.

Any integration over angle becomes a quadrature sum, so that
$$ \int_{-1}^{1} P_{l} (\mu') \psi(x,\mu')  \ud\mu'
\lra
\sum_{n'=1}^{N} w_{n'} P_{l} (\mu_n') \psi_{n'}(x) \,.$$
The S$_\mathrm{N}$ equations are, modifying Eq.~\eqref{eq:exact},
\begin{equation}
\mu_n \oder{\psi_n}{x}(x) + \Sigma_t(x) \psi_n(x) 
= \sum_{l=0}^{L}  \frac{2l+1}{2} \Sigma_{sl}(x) P_{l}(\mu_n)
  \sum_{n'=1}^{N} w_{n'} P_{l} (\mu_n') \psi_{n'}(x)
  + Q_n(x)
  \label{eq:sn}
\end{equation}
for $0 \le x \le X$ and $n=1,\ldots,N$.

An isotropic source should be 
\begin{equation}
Q_n(x) = \frac{1}{2} Q(x) \,.
  \label{eq:isotropicsource}
\end{equation}
Additionally, boundary conditions such as those in
Eqs.~\eqref{eqs:exactwithbc} are required. I have not yet looked into the
evaluation of discrete boundary conditions for an arbitrary incident flux, but
the reflecting boundary conditions $\psi(0, \mu) = \psi(0, -\mu)$ for $0 < \mu
< 1$ have their obvious counterpart in the \SN{} equations, for example:
$$ \psi_{n}(0) = \psi_{N + 1 - n}(0) \,, \quad N/2 + 1 < n < N  \,.$$

For vacuum boundaries, of course, $\psi_{n}(0) = 0$ for the incoming directions.
%%%%%%%%%%%%%%%%%%%%%%%%%%%%%%%%%%%%%%%%%%%%%%%%%%%%%%%%%%%%%%%%%%%%%%%%%%%%%%%
\subsection{The $k$-eigenvalue problem}
In the eigenvalue problem, the source $Q$ is replaced with a fission production
term, the product of the fission cross section, the number of neutrons emitted
per fission, and the scalar flux. (This gives the space-dependent production
rate of fission neutrons.)
It is scaled by $1/k$, a value which balances the production and loss
terms in the equation. Fission neutrons are always assumed to be emitted
isotropically, and since this is a one-group problem, $\chi = 1$. 
\begin{multline}
  \mu_n \oder{\psi_n}{x}(x) + \Sigma_t(x) \psi_n(x) 
\\
  = \sum_{l=0}^{L}  \frac{2l+1}{2} \Sigma_{sl}(x) P_{l}(\mu_n)
  \sum_{n'=1}^{N} w_{n'} P_{l} (\mu_n') \psi_{n'}(x)
  + \frac{1}{k} \nu\Sigma_f(x) \sum_{n'=1}^{N} w_{n'} \psi_{n'}(x)
  \label{eq:fissionsn}
\end{multline}

In operator notation, Eq.~\eqref{eq:fissionsn} can be
written as
\begin{equation}
\op{L} \psi = \op{M} \op{S} \op{D} \psi +  \frac{1}{k} \op{F} \op{D} \psi \,,
  \label{eq:fissionoperator}
\end{equation}

where $\op{L}$ is the loss (leakage and absorption) operator; 
$\op{S}$ is the scattering operator, which works on the Legendre moments of
the angular flux; $\op{D}$ is the ``discrete-to-moments'' operator, which
integrates
the different moments of the discrete moments angular flux (mapping the
discretized angular flux to Legendre moments); $\op{M}$ is the
``moments-to-discrete'' operator; and $\op{F}$ is the
fission production operator. Moving the scattering operator to the left, and
inverting the transport operator,
$$ k \psi = (\op{L} - \op{M} \op{S} \op{D} )^{-1}  (\op{F} \op{D}) \psi \,,$$
is easily recognized as an eigenvalue problem.

[Note: in most computer programs, the full angular flux is never stored due to
memory requirements. Instead, only the scalar flux (and higher moments if
anisotropic scattering is present) is stored, using $\phi = \op{D} \psi$. Then,
when the transport sweep is performed, the flux moments are expanded via the
$\op{D}$ moment-to-discrete operator. Eq.~\eqref{eq:fissionoperator} is then
usually
\begin{align*}
\op{L} \psi &= \op{M} \op{S} \op{D} \psi +  \frac{1}{k} \op{F} [\op{D} \psi]
\\
( \op{L} -  \op{M} \op{S} \op{D} ) \psi &=  \frac{1}{k} \op{F} \phi
\\
k \psi &=  ( \op{L} - \op{M} \op{S} \op{D} ) ^{-1} \op{F} \phi
\\
k \op{D} \psi &= \op{D}  ( \op{L} - \op{M} \op{S} \op{D} ) ^{-1} \op{F} \phi
\\
k \phi &= \op{D} ( \op{L} - \op{M} \op{S} \op{D} ) ^{-1} \op{F} \phi
\end{align*}
See Ref.~\cite{War2004a} for more information on the detailed eigenvalue
problem.]

%%%%%%%%%%%%%%%%%%%%%%%%%%%%%%%%%%%%%%%%%%%%%%%%%%%%%%%%%%%%%%%%%%%%%%%%%%%%%%%
\subsection{Infinite medium}
The infinite-medium homogeneous \SN{} equations with an isotropic source are
$$\Sigma_t \psi_n 
= \sum_{l=0}^{L}  \frac{2l+1}{2} \Sigma_{sl} P_{l}(\mu_n)
  \sum_{n'=1}^{N} w_{n'} P_{l} (\mu_n') \psi_{n'}
  + \frac{1}{2} Q_0 \,.$$

We multiply each equation by its corresponding weight, and add them all
together:
\begin{equation}
\Sigma_t \sum_{n=1}^{N} w_{n} \psi_{n} 
= \sum_{n=1}^{N} w_{n} \sum_{l=0}^{L}  \frac{2l+1}{2} \Sigma_{sl} P_{l}(\mu_n)
  \sum_{n'=1}^{N} w_{n'} P_{l} (\mu_n') \psi_{n'}
  + \sum_{n=1}^{N} w_{n} \frac{1}{2} Q_0
  \label{eq:addedinf}
\end{equation}
The scalar flux is the zeroth Legendre moment of the equation ($P_l(\mu) = 1$),
so
$$ \phi(x) = \sum_{n=1}^{N} w_{n} \psi_{n}(x) \,.$$
Then, Eq.~\eqref{eq:addedinf} is:
$$ \Sigma_t [\phi] 
= \sum_{l=0}^{L}  \frac{2l+1}{2} \Sigma_{sl} \left[ \sum_{n=1}^{N} w_{n}
P_{l}(\mu_n) \right] \left[ \sum_{n'=1}^{N} w_{n'} P_{l} (\mu_n') \psi_{n'}
\right] + \frac{1}{2} Q_0 [2] $$

If the Gauss-Legendre quadrature set is used, then from
\cite[p.121]{Lew1984}:
$$ \sum_{n=1}^{N} w_{n} P_{l}(\mu_n) = 2 \delta_{l0} \,, l = 0,\ldots, N-1 \,.$$
In this case, the scattering term collapses and the equation becomes
\begin{align*}
  \Sigma_t \phi 
  &= \sum_{l=0}^{L}  \frac{2l+1}{2} \Sigma_{sl} \left[ 2 \delta_{l0} \right] \left[ \sum_{n'=1}^{N} w_{n'} P_{l} (\mu_n') \psi_{n'}
\right] + Q_0
\\
  \Sigma_t \phi 
  &=  \frac{1}{2} \Sigma_{s0} (2) \left[ \sum_{n'=1}^{N} w_{n'} P_{0} (\mu_n') \psi_{n'}
\right] + Q_0
\\
\Sigma_t \phi 
  &=  \Sigma_{s0} \phi + Q_0
\\
\phi 
&=  \frac{Q_0}{\Sigma_t - \Sigma_{s0}} \,.
\end{align*}
This is the well-known infinite medium result.
%%%%%%%%%%%%%%%%%%%%%%%%%%%%%%%%%%%%%%%%%%%%%%%%%%%%%%%%%%%%%%%%%%%%%%%%%%%%%%%%
\section{Spatial discretizations: finite difference}
We impose a spatial grid with $I+1$ points spanning $0 \le x \le X$. The
space-dependent cross sections and source are assumed to be piecewise-constant
inside the grid: for example, $\Sigma_t(x) = \Sigma_{t,i+1/2}$ for $x_{i} < x
< x_{i+1}$.

With finite difference equations, we integrate the \SN{} equations over each
spatial interval, operating on Eq.~\eqref{eq:sn} by $\int_{x_{i}}^{x_{i+1}}
(\cdot) \ud x$ for $i = 0,1,\ldots, I-1$.
This results in $I$ equations for each of the $N$ directions (so $I \cdot N$
total equations). Letting $\psi_{n,i} = \psi_n(x_i)$, we have
\begin{multline}
  \mu_n \left[ \psi_{n,i+1} - \psi_{n,i} \right] + \Sigma_{t,i+1/2}
  \int_{x_{i}}^{x_{i+1}} \psi_n(x) \ud x
\\
= \sum_{l=0}^{L}  \frac{2l+1}{2} \Sigma_{sl,i+1/2} P_{l}(\mu_n)
 \sum_{n'=1}^{N} w_{n'} P_{l} (\mu_n') \int_{x_{i}}^{x_{i+1}} \psi_{n'}(x) \ud x
  + Q_{n,i+1/2} \int_{x_{i}}^{x_{i+1}} \ud x
  \label{eq:finitediff}
\end{multline}

The \emph{diamond difference} discretization essentially assumes that the
angular flux between two grid points is the average of those two, so the
essential approximation is that
\begin{equation}
\int_{x_{i}}^{x_{i+1}} \psi_{n}(x) \ud x
\approx \psi_{n,i+1/2} \int_{x_{i}}^{x_{i+1}}  \ud x 
\approx  \left[ \frac{\psi_{n,i+1} + \psi_{n,i}}{2} \right] [x_{i+1} - x_{i}]
\,.
  \label{eq:diamondapprox}
\end{equation}
The width of each spatial cell is $x_{i+1} - x_{i} \equiv \Delta_i$. From this
point, we make the substitution $X_{i+1/2} \to X_{i}$ for brevity, keeping in
mind that the cross section or source for index $i$ refers to the cell between
$x_i$ and $x_{i+1}$.

We have a total of $N\cdot(I+1)$ unknowns: the angular flux in every
direction $N$ on every grid point. The missing equations are resolved by the
boundary condition: on the left boundary ($i=0$) the angular flux $\phi_{n,0}$
for incoming directions ($\mu_n > 0$) must be specified, and likewise on the
right boundary ($i=I$) the angular flux $\phi_{n,I}$ must be specified
for directions with $\mu_n < 0$.

For isotropic scattering, Eq.~\eqref{eq:finitediff} has $L=0$, and simplifies to
$$
 \mu_n \left[ \psi_{n,i+1} - \psi_{n,i} \right] + \Sigma_{t,i}
  \int_{x_{i}}^{x_{i+1}} \psi_n(x) \ud x
= \frac{1}{2} \Sigma_{s0,i} 
 \sum_{n'=1}^{N} w_{n'}  \int_{x_{i}}^{x_{i+1}} \psi_{n'}(x) \ud x
  + Q_{n,i}  \Delta_i
$$
Substituting Eq.~\eqref{eq:diamondapprox},
$$ 
\mu_n \left[ \psi_{n,i+1} - \psi_{n,i} \right]
+ \Sigma_{t,i} \Delta_i \left[ \frac{\psi_{n,i+1} + \psi_{n,i}}{2} \right]
=
\frac{1}{2} \Sigma_{s0,i} \Delta_i 
\sum_{n'=1}^{N} w_{n'} \left[ \frac{\psi_{n',i+1} + \psi_{n',i}}{2} \right]
  + Q_{n,i} \Delta_i
$$

Let $r = iN + n$, each equation for $r = 1, \ldots, IN$ is:
$$
\mu_n \left[ \psi_{r + N} - \psi_{r} \right]
+ \Sigma_{t,i} \Delta_i \left[ \frac{\psi_{r + N} + \psi_{r}}{2} \right]
=
\frac{1}{2} \Sigma_{s0,i} \Delta_i 
\sum_{n'=1}^{N} w_{n'} \left[ \frac{\psi_{n',i+1} + \psi_{n',i}}{2} \right]
  + Q_{r} \Delta_i
$$

%%%%%%%%%%%%%%%%%%%%%%%%%%%%%%%%%%%%%%%%%%%%%%%%%%%%%%%%%%%%%%%%%%%%%%%%%%%%%%%%
\nocite{Lew1984,Lar2007}
\bibliographystyle{siam}
\bibliography{references}
\end{document}
